\stockustrade
\setbleed{.125in}
\setlrmarginsandblock{.75in}{*}{*}
\setulmarginsandblock{.75in}{.75in}{*}
\checkandfixthelayout
\typeoutlayout

\usepackage{polyglossia}
% \setdefaultlanguage{english}
\usepackage[useregional]{datetime2}

% extract info from git commit
% We override gitinfo2.sty's constant because we don't use the git hooks;
% instead the compile script extracts the info
\makeatletter\newcommand{\GI@githeadinfo@file}{gitHeadInfo.gin}\makeatother
\usepackage[dirty=*]{gitinfo2}
\usepackage{xstring}% \IfEq and \StrCut
% a couple extensions...
\makeatletter
\newcommand\gitCommitInfo{%
  \IfEq{\gitRel}{}
    {commit \gitAbbrevHash\gitDirty}
    {\IfEq{\gitRoff}{0}
      {release \gitRel}
      {modified since release \gitRel{} --- commit \gitAbbrevHash\gitDirty}}}
% gitinfo2 only defines \gitdate for datetime.sty, not datetime2
\@ifpackageloaded{datetime2}{%
    \IfEq{\gitAuthorDate}{\gitInf@missing}{%
        \DTMsavenoparsedate{gitdate}{\@dtm@currentyear}{\@dtm@currentmonth}{\@dtm@currentday}{\@dtm@currentdow}
    }{%
        \StrCut{\gitAuthorDate}{-}{\gitInf@year}{\gitInf@md}
        \StrCut{\gitInf@md}{-}{\gitInf@month}{\gitInf@day}
        \DTMsavenoparsedate{gitdate}{\gitInf@year}{\gitInf@month}{\gitInf@day}{-1}
    }
    \newcommand\gitdate{\DTMusedate{gitdate}}
}{}%
\makeatother

\usepackage[normalem]{ulem} % for strikeout text (\sout{...})
\usepackage{graphicx}

\usepackage[% make hyperlinks look like normal text
  colorlinks=true,
  linkcolor=black,
  urlcolor=black,
  citecolor=black
]{hyperref}

% Lulu only supports PDF format version 1.3
\pdfminorversion=3

\usepackage{multicol}

% a nice horizontal separator
\newcommand{\sectionline}{%
  \begin{center}\nointerlineskip
    \rule{0.233\linewidth}{.7pt}%
  \end{center}
}

% less space around lists
% \firmlists
\tightlists

% showing code, inline and as blocks
\usepackage{listings}
\usepackage{lstsmalltalk}
\usepackage{lsthttp}
\lstdefinelanguage{plain}{% verbatim, plain text
}
\lstalias{CSS}{plain}
\lstalias{STON}{plain}
\lstalias{Javascript}{plain}

\lstset{language=smalltalk}
\newcommand\code[1]{\texttt{\textup{#1}}}
\lstnewenvironment{listing}[1][]{%
  \lstset{%
    frame=none,
    mathescape=false,
    tabsize=2,
    keywordstyle={},
    stringstyle={},
    commentstyle={},
    #1
  }%
}{}

% highlighted paragraphs
\newenvironment{note}{%
  \begin{leftbar}\textbf{Note}\quad
}{\end{leftbar}}

\newenvironment{todo}{%
  \begin{leftbar}\textbf{To do}\quad
}{\end{leftbar}}
